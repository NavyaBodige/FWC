\documentclass{article}
\usepackage{setspace}
\usepackage{gensymb}
\usepackage{xcolor}
\usepackage{caption}
%\usepackage{subcaption}
%\doublespacing
\singlespacing 
%\usepackage{amssymb}
%\usepackage{relsize}
\usepackage[cmex10]{amsmath}
\usepackage{mathtools}
%\usepackage{amsthm}
%\interdisplaylinepenalty=2500
%\savesymbol{iint}
%\usepackage{txfonts}
%\restoresymbol{TXF}{iint}
%\usepackage{wasysym}
\usepackage{amsthm}
\usepackage{mathrsfs}
\usepackage{txfonts}
%\usepackage{stfloats}
\usepackage{float}
\usepackage{cite}
\usepackage{cases}
\usepackage{subfig}
%\usepackage{xtab}
\usepackage{longtable}
\usepackage{multirow}
%\usepackage{algorithm}
%\usepackage{algpseudocode}
\usepackage{enumitem}
\usepackage{mathtools}
%\usepackage{eenrc}
%\usepackage[framemethod=tikz]{mdframed}
\usepackage{listings}
\usepackage{listings} 
\usepackage[latin1]{inputenc} 
%% \usepackage{color} 
%% \usepackage{array} 
%% \usepackage{longtable} 
%% \usepackage{calc} 
%% \usepackage{multirow} 
%% \usepackage{hhline} 
%% \usepackage{ifthen} 
%% %optionally (for landscape tables embedded in another document): 
%% \usepackage{lscape} 
\usepackage{titling}
%\usepackage{fulbigskip}
\usepackage{tikz}
\usepackage{graphicx}
\graphicspath{{/Internal storage/Download/fwc/figs}}
\begin{document} 
\title{CLASS 9\\10.CIRCLES}
\date{}
\maketitle
\section{EXERCISE 1}
\begin{enumerate}
\item $AD$ is a diameter of a circle and $AB$ is a chord. If $AD = 34 cm$, $AB = 30 cm$, the distance of $AB$ from the centre of the circle is:
\begin{enumerate}
\item 17cm
\item 15cm
\item 4cm
\item 8cm
\end{enumerate}
\item In (Figure \ref{fig:10.3}), if $OA = 5cm$, $AB = 8cm$ and $OD$ is perpendicular to $AB$, then $CD$ is equal to:
\begin{figure}[H]
\centering
\includegraphics[width=\columnwidth]{figs/10.3.jpg}
\caption{}
\label{fig:10.3}
\end{figure}
\begin{enumerate}
\item 2cm
\item 3cm
\item 4cm
\item 5cm
\end{enumerate}
\item If $AB = 12 cm$, $BC = 16 cm$ and $AB$ is perpendicular to $BC$, then the radius of the circle passing through the points $A$, $B$ and $C$ is:
\begin{enumerate}
\item 6cm
\item 8cm
\item 10cm
\item 12cm
\end{enumerate}
\item In (Figure \ref{fig:10.4}), if $\angle$ $ABC = 20$$^{\circ}$ , then $\angle$$AOC$ is equal to: 
\begin{figure}[H]
\centering
\includegraphics[width=\columnwidth]{figs/10.4.jpg}
\caption{}
\label{fig:10.4}
\end{figure}
\begin{enumerate}
\item 20$^{\circ}$
\item 40$^{\circ}$
\item 60$^{\circ}$
\item 10$^{\circ}$
\end{enumerate}
\item In (Figure \ref{fig:10.5}), if $AOB$ is a diameter of the circle and $AC = BC$,then $\angle$$CAB$ is equal to:
\begin{figure}[H]
\centering
\includegraphics[width=\columnwidth]{figs/10.5.jpg}
\caption{}
\label{fig:10.5}
\end{figure}
\begin{enumerate}
\item 30$^{\circ}$
\item 60$^{\circ}$
\item 90$^{\circ}$
\item 45$^{\circ}$
\end{enumerate}
\item In (Figure \ref{fig:10.6}), if $\angle$$OAB = 40$$^{\circ}$ , then $\angle$$ACB$ is equal to:         
\begin{figure}[H]
\centering
\includegraphics[width=\columnwidth]{figs/10.6.jpg}
\caption{}
\label{fig:10.6}
\end{figure}
\begin{enumerate}
\item 50$^{\circ}$
\item 40$^{\circ}$
\item 60$^{\circ}$
\item 70$^{\circ}$
\end{enumerate}
\item In (Figure \ref{fig:10.7}), if $\angle$$DAB = 60$$^{\circ}$ , $\angle$$ABD = 50$$^{\circ}$ , then $\angle$$ACB$ is equal to:
\begin{figure}[H]
\centering
\includegraphics[width=\columnwidth]{figs/10.7.jpg}
\caption{}
\label{fig:10.7}
\end{figure}
\begin{enumerate}
\item 60$^{\circ}$
\item 50$^{\circ}$
\item 70$^{\circ}$
\item 80$^{\circ}$
\end{enumerate}
\item $ABCD$ is a cyclic quadrilateral such that $AB$ is a diameter of the circle circumscribing it and $\angle$$ADC = 140$$^{\circ}$ , then $\angle$$BAC$ is equal to:
\begin{enumerate}
\item 80$^{\circ}$
\item 50$^{\circ}$
\item 40$^{\circ}$
\item 30$^{\circ}$
\end{enumerate}
\item In (Figure \ref{fig:10.8}), $BC$ is a diameter of the circle and $\angle$$BAO = 60$$^{\circ}$. Then $\angle$ADC is equal to:
\begin{figure}[H]
\centering
\includegraphics[width=\columnwidth]{figs/10.8.jpg}
\caption{}
\label{fig:10.8}
\end{figure}
\begin{enumerate}
\item 30$^{\circ}$
\item 45$^{\circ}$
\item 60$^{\circ}$
\item 120$^{\circ}$
\end{enumerate}
\item In (Figure \ref{fig:10.9}), $\angle$$AOB = 90$$^{\circ}$ and $\angle$$ABC = 30$$^{\circ}$ , then $\angle$$CAO$ is equal to:            \begin{figure}[H]
\centering
\includegraphics[width=\columnwidth]{figs/10.9.jpg}
\caption{}
\label{fig:10.9}
\end{figure}
\begin{enumerate}
\item 30$^{\circ}$
\item 45$^{\circ}$
\item 90$^{\circ}$
\item 60$^{\circ}$
\end{enumerate}
\end{enumerate}
\end{document}
